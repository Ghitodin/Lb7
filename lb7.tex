\documentclass{article}
\usepackage[a4paper, margin=1in]{geometry}
\usepackage[utf8]{inputenc}
\usepackage[T2A]{fontenc}
\usepackage{tcolorbox}
\usepackage{listings}
\usepackage{multicol}

\lstdefinestyle{mystyle}{
    backgroundcolor=\color{gray!20},   
    commentstyle=\color{codegreen},
    keywordstyle=\color{magenta},
    numberstyle=\tiny\color{codegray},
    stringstyle=\color{codepurple},
    basicstyle=\ttfamily\footnotesize,
    breakatwhitespace=false,         
    breaklines=true,                 
    captionpos=b,                    
    keepspaces=true,                 
    numbers=left,                    
    numbersep=5pt,                  
    showspaces=false,                
    showstringspaces=false,
    showtabs=false,                  
    tabsize=2
}

\lstset{style=mystyle}

\begin{document}
\begin{center}
    \LARGE Тест з програмування на Python \\
    \vspace{1em}
    Ім'я: \underline{\hspace{5cm}} Призвище: \underline{\hspace{5cm}} \\
    \vspace{1em}
    Що буде виведено на екран в результаті виконання наступних команд?
\end{center}

\begin{multicols}{2}
\begin{tcolorbox}[rounded corners]
\textbf{Завдання 1.} Що виведе наступний код:
\begin{lstlisting}[language=Python]
s = 0
for k in range(3, 11):
    s = s + k
print(s)
\end{lstlisting}
Варіанти відповіді: \\
1. 45 \\
2. 52 \\
3. 25 \\
4. 14 \\
\end{tcolorbox}

\begin{tcolorbox}[rounded corners]
\textbf{Завдання 2.} Що виведе наступний код:
\begin{lstlisting}[language=Python]
z = 5
for n in range(5):
    if n > 10:
        z = z - n
    else:
        z = z + n
print(z)
\end{lstlisting}
Варіанти відповіді: \\
1. 15 \\
2. 51 \\
3. 0 \\
4. 10 \\
\end{tcolorbox}

\begin{tcolorbox}[rounded corners]
\textbf{Завдання 3.} Що виведе наступний код:
\begin{lstlisting}[language=Python]
a = 15
b = 5
while a > b:
    if a % 2 == 0:
        b = b + a
    else:
        a = a - 2 * b + 1
print(b)
\end{lstlisting}
Варіанти відповіді: \\
1. 20 \\
2. 6 \\
3. 11 \\
4. 5 \\
\end{tcolorbox}

\begin{tcolorbox}[rounded corners]
\textbf{Завдання 4.} Що виведе наступний код:
\begin{lstlisting}[language=Python]
c = 0
m = 8
while m > 1:
    d = m % 10
    c = (c + d) * 10
    m = m // 10
print(c)
\end{lstlisting}
Варіанти відповіді: \\
1. 20 \\
2. 8 \\
3. 80 \\
4. 0.8 \\
\end{tcolorbox}

\begin{tcolorbox}[rounded corners]
\textbf{Завдання 5.} Що виведе наступний код:
\begin{lstlisting}[language=Python]
n = 2
v = 4
r = 5
while True:
    n = n + n * v
    r += 1
    if n >= 20:
        break
print(r)
\end{lstlisting}
Варіанти відповіді: \\
1. 8 \\
2. 6 \\
3. 5 \\
4. 7 \\
\end{tcolorbox}
\end{multicols}

\end{document}
